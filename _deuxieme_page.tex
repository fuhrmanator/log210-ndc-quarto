\small
\begin{description}
\item[Ce manuel] est utilisé dans le cadre du cours \emph{Analyse et
conception de logiciels (LOG210)} à l'École de technologie supérieure
(ÉTS).
\item[Les auteurs]sont Christopher Fuhrman et Yvan Ross.\\
Christopher Fuhrman (page \href{https://github.com/fuhrmanator}{GitHub} et \href{https://scholar.google.com/citations?user=Qa5SpUwAAAAJ\&hl=fr\&oi=ao}{Google Scholar}) est professeur à l'ÉTS depuis 2001.
Son domaine de recherche est la conception de logiciels.\\
Yvan Ross (page \href{https://github.com/yvanross}{GitHub}) est chargé de cours à l'ÉTS depuis 2013. Il a enseigné LOG210 de nombreuses fois et a contribué à la création d'exercices et aux travaux de laboratoire utilisés dans le cadre de ce cours.
\end{description}

%%% Remerciements
\textbf{Remerciements aux personnes suivantes~:} Conversion markdown~: Clyde Vianney Omog Ntap\,; Révision~: Katerine Robert\,; Relecture~: Olivier Brochu, Roberto Erick Lopez-Herrejon, Mouna Moumene, Taki Eddine Seghiri. 

%%% Support fabriqueREL

Ce manuel a été réalisé avec le soutien de la
fabriqueREL. Fondée en 2019, la fabriqueREL est
portée par divers établissements d’enseignement
supérieur du Québec et agit en collaboration avec
les services de soutien pédagogique et les
bibliothèques. Son but est de faire des ressources
éducatives libres (REL) le matériel privilégié en
enseignement supérieur au Québec.

% \vfill
\begin{tcolorbox}[
	colback=white, 
	colframe=black,
	toprule=1pt,
	bottomrule=1pt,
	leftrule=1pt,
	rightrule=1pt,
	before upper={\begin{minipage}[t]{\linewidth}},
    after upper={\end{minipage}},
]
\begin{tabular}{m{0.25\linewidth}m{0.7\linewidth}}
\includegraphics{images/CC/by.eps} & 
Sauf indications contraires, le contenu de ce manuel électronique est disponible en vertu des termes de la \href{https://creativecommons.org/licenses/by/4.0/deed.fr}{Licence Creative Commons Attribution 4.0 International} \\
\end{tabular}
{\footnotesize
\begin{description}
  \item[Vous êtes autorisé à~:] ~\\
  \textbf{Partager} -- copier, distribuer et communiquer le matériel par tous moyens et sous tous formats.\\
  \textbf{Adapter} -- remixer, transformer et créer à partir du matériel pour toute utilisation, y compris commerciale.

  % L'Offrant ne peut retirer les autorisations concédées par la licence
% tant que vous appliquez les termes de cette licence.

\item[Selon les conditions suivantes~:] ~\\
  \textbf{Attribution} -- Vous devez \href{https://creativecommons.org/licenses/by/4.0/deed.fr\#}{créditer} l'Œuvre, intégrer un lien vers la licence et indiquer si des modifications ont été effectuées à l'Œuvre. Vous devez indiquer ces informations par tous les moyens raisonnables, sans toutefois suggérer que l'Offrant vous soutient ou soutient la façon dont vous avez utilisé son Œuvre.\\
  \textbf{Pas de restrictions complémentaires} -- Vous n'êtes pas autorisé à appliquer des conditions légales ou des \href{https://creativecommons.org/licenses/by/4.0/deed.fr\#}{mesures techniques} qui restreindraient légalement autrui à utiliser l'Œuvre dans les conditions décrites par la licence
\end{description}
}
\end{tcolorbox}

% \textbf{Notes~:}

% Vous n'êtes pas dans l'obligation de respecter la licence pour les éléments ou matériel appartenant au domaine public ou dans le cas où l'utilisation que vous souhaitez faire est couverte par une \href{https://creativecommons.org/licenses/by/4.0/deed.fr\#}{exception}.
% Aucune garantie n'est donnée. Il se peut que la licence ne vous donne pas toutes les permissions nécessaires pour votre utilisation. Par exemple, certains droits comme \href{https://creativecommons.org/licenses/by/4.0/deed.fr\#}{les droits moraux, le droit des données personnelles et le droit à l'image} sont susceptibles de limiter votre utilisation.

ISBN~: \\
Dépôt légal 2023 \\
DOI~: \\
Handle (permalien)~: \\
© Christopher Fuhrman et Yvan Ross (2023)\\
Pour citer cet ouvrage~: Christopher Fuhrman et Yvan Ross. (2023). Analyse et conception de logiciels.
École de technologie supérieure. \href{https://creativecommons.org/licenses/by/4.0/deed.fr}{CC BY}.

\vglue5mm

\vfill
\begin{tabular}{m{3.5in}m{3.5in}}
\includegraphics[height=1.5in]{images/F-REL_logo-coul-horiz.png} &
\includegraphics[width=1.5in]{images/Logo_ETS_TypoGrise_D_FR.eps}\\
\end{tabular}
% \normalsize